\documentclass[10pt,a4paper]{article}
\setlength{\topmargin}{-5mm}
\setlength{\textheight}{244mm}
\setlength{\oddsidemargin}{0mm}
\setlength{\textwidth}{165mm}

\setlength{\parindent}{0pt}

\usepackage{fancyvrb}
\DefineVerbatimEnvironment{verbatim}{Verbatim}{xleftmargin=4mm}

\usepackage{lmodern}
\usepackage[T1]{fontenc}
\usepackage{pxfonts}
\usepackage{textcomp}

\usepackage[square]{natbib}
\citestyle{nature}

\usepackage{listings}
\lstset{basicstyle=\ttfamily,
escapeinside={||}}

\frenchspacing

\title{Entry}
\author{Jeremy Mates}
\date{June 26, 2021}

\usepackage{hyperref}
\hypersetup{pdfauthor={Jeremy Mates},pdftitle={Entry}}

\begin{document}
\renewcommand{\thefootnote}{\alph{footnote}}
\bibliographystyle{plainnat}
\maketitle

\begin{lstlisting}
$ |\textbf{FOO=a ruby \symbol{45}e 'puts ENV\symbol{91}"FOO"\symbol{93}, ARGV' b c}|
a
b
c
\end{lstlisting}

An important difference between the arguments list--\texttt{b} and
\texttt{c} to the ruby script, or a longer list at the \texttt{exec}
level--and the environment list--\texttt{FOO=a} and various unseen
others--is that environment variables are inherited by child processes.
This means placing a password or other secret data in an environment
variable may be a poor practice. Environment variables can be difficult
to whitelist, see \texttt{sudo} or \texttt{doas} for example code. \\

Use the \texttt{env(1)} command to inspect what environment variables
are set, or language-specific features to view what is often the same
list. Here are some different ways of doing the same thing.

\begin{lstlisting}
$ |\textbf{export FOO=zot}|
$ |\textbf{expect \symbol{45}c 'puts \symbol{36}env(FOO); puts \symbol{91}exec env \symbol{124} grep FOO\symbol{93}'}|
zot
FOO=zot
$ |\textbf{perl \symbol{45}E 'say \symbol{36}ENV\symbol{123}FOO\symbol{125}; exec "env"' \symbol{124} grep zot}|
zot
FOO=zot
$ |\textbf{ruby \symbol{45}e 'puts ENV\symbol{91}"FOO"\symbol{93}; exec "env"' \symbol{124} grep zot}|
FOO=zot
\end{lstlisting}

Note that the \texttt{expect} (TCL) \texttt{exec} is nothing like the OS
\texttt{execv(3)} system call or the Perl or Ruby \texttt{exec}
functions; the TCL \texttt{exec} is actually what is usually called
\texttt{system(3)}. \texttt{system(3)} runs a shell command while the
\texttt{execv(3)} replaces the current process with something else.
Which to use depends on the language, and whether you want the original
process to hang around afterwards. \texttt{system(3)} is often exploited
if an attacker can pass in arbitrary shell code, somehow. \\

\texttt{ruby} needs a \texttt{STDOUT.flush} call to behave like the
other processes; this is a buffering issue.

\begin{lstlisting}
$ |\textbf{ruby \symbol{45}e 'puts ENV\symbol{91}"FOO"\symbol{93}; STDOUT.flush; exec "env"' \symbol{124} grep zot}|
zot
FOO=zot
\end{lstlisting}

On linux, the \texttt{/proc/.../environ} file is another way to inspect
the environment of a process.

\begin{lstlisting}
# |\textbf{cd /proc/\symbol{36}\symbol{36}}|
# |\textbf{cat environ}|
USER=rootLOGNAME=rootHOME=/rootPATH=/usr/local/sbin:/us...
\end{lstlisting}

If the output is a mess, run it through a hex
viewer--\texttt{hexdump(1)}, \texttt{od(1)} or with \texttt{vim}
installed \texttt{xxd}--to see what is going on. This may reveal
characters such as

\begin{lstlisting}
# |\textbf{hexdump \symbol{45}C environ \symbol{124} head \symbol{45}1}|
00000000  55 53 45 52 3d 72 6f 6f  74 00 4c 4f ...
\end{lstlisting}

\texttt{nul} as shown by the hex code \texttt{00}. Now is a good time to
mention \texttt{ascii(7)} which contains a handy chart of the ASCII
characters; Unicode or other not-ASCII encodings can (sometimes) be
identified by hex values north of \texttt{7F} and may need
encoding-aware tools to better display and debug them. The
\texttt{tr(1)} command should suffice to improve \texttt{environ} for
human consumption; map the \texttt{NUL} values to newlines:

\begin{lstlisting}
# |\textbf{tr '\symbol{92}0' '\symbol{92}n' < environ}|
...
\end{lstlisting}

Various invisible characters--or Unicode characters that appear to be
similar to an ASCII hyphen but are not--can also be problematic. A hex
viewer should reveal their presence. The font will also influence how
similar different characters appear to be. Even worse, terminals or
operating systems may normalize characters run through copy and paste,
or a document may contain something besides the indicated character. All
this can complicate debugging.

\begin{lstlisting}
$ |\textbf{ls --al}|
|ls: --al: No such file or directory|
$ |\textbf{echo 'ls --al' \symbol{124} hexdump}|
00000000 6c 73 20 e2 80 93 61 6c 0a
\end{lstlisting}

Here the \texttt{e2 80 93} between the space \texttt{20} and the
letter a \texttt{61} represent the Unicode character 'EN DASH'
\texttt{U+2013} in UTF-8. \\

When something weird happens, run the data through a hex viewer, or use
a process tracing tool such as \texttt{strace}. \texttt{strace} will
reveal the exact argument list used and may help diagnose how an unknown
ball of code is calling external programs. This usually boils down to
some \texttt{exec} call, though there may also be calls to \texttt{sh}
involved, depending on the input.

\begin{lstlisting}
$ |\textbf{strace \symbol{45}f  \symbol{45}o eg1 ruby \symbol{45}e '`echo easygrep`'}|
$ |\textbf{strace \symbol{45}f  \symbol{45}o eg2 ruby \symbol{45}e '`echo easygrep\symbol{38}`'}|
$ |\textbf{grep easygrep eg* \symbol{124} grep execve \symbol{124} grep \symbol{45}v ruby}|
|eg1:4186 execve("/usr/bin/echo", \symbol{91}"echo", "easygrep"\symbol{93} ...|
|eg2:4206 execve("/bin/sh", \symbol{91}"sh", "\symbol{45}c", "echo easygrep\symbol{38}"\symbol{93} ...|
\end{lstlisting}

With a single character change \texttt{ruby} switches from performing a
direct exec of \texttt{echo} to wrapping the command in a shell call.
This may be good to know if one had assumed the command was always run
under the shell, or the other way around, and one or the other methods
is broken for some reason. There can also be security implications.

\section*{On the Ordering of Arguments}

Another complication with arguments is that programs may support
subcommands where the arguments for one level of the command probably
should not be mixed with those of the subcommand; other programs such as
\texttt{find(1)} or \texttt{pw} on FreeBSD may require that various
arguments appear in specific ways;

\begin{lstlisting}
# wrong!
|find \symbol{45}name "foo*" /etc|

# correct
|find /etc \symbol{45}name "foo*"|

# same -C, different meanings
|git \symbol{45}C <path> commit \symbol{45}C <commit> ...|
\end{lstlisting}

this is in contrast to other commands where the options can actually
appear after other arguments. Such usages may be frowned on, or
unportable.

\begin{lstlisting}
# atypical
$ |\textbf{grep root /etc \symbol{45}r}|
...

# more common
$ |\textbf{grep -r root /etc}|
...

# disable option processing, see getopt(3)
$ |\textbf{grep \symbol{45}r \symbol{45}\symbol{45} "\symbol{36}pattern" /etc}|
...
\end{lstlisting}

In general the manual page must be studied to see which inputs and in
what order are supported; keep in mind that there are various historical
warts, \texttt{find(1)} and especially \texttt{dd(1)} that use distinct
grammars or otherwise do not fit the mold of more common programs. \\

Now let us take a closer look at environment variables, and how to

\section*{Duplicate Environment Variables}

\ldots but aren't environment variables stored in some sort of hash or
associative array, which cannot by definition have duplicate keys?
Yes. In some languages. Which are abstractions on top of what is
actually a list, and lists may contain duplicate entries. Also there's
literally nothing that distinguishes command line arguments from the
environment variables. Some assembly and the judicious use of a
debugger may help clear up this fog, or at least exchange the fog for
the deep end of the pool.

\lstinputlisting{code/envandargs.asm}

This is NASM format assembly for an amd64 Linux system that does little
more than pop 64-bit values off the stack into a register. This is only
useful under a debugger. With this file saved as \texttt{envandargs.asm}
the assembly, linking, and launching it under a debugger may run
something like

\begin{lstlisting}
$ |\textbf{nasm \symbol{45}f elf64 \symbol{45}g \symbol{45}F dwarf \symbol{45}o envandargs.o envandargs.asm}|
$ |\textbf{ld \symbol{45}o envandargs envandargs.o}|
$ |\textbf{gdb envandargs}|
...
\end{lstlisting}

Then, in \texttt{gdb}, we keep track of the \texttt{RCX} register:

\begin{lstlisting}
|(gdb) \textbf{set args a bb ccc}|
|(gdb) \textbf{display /x \symbol{36}rcx}|
|(gdb) \textbf{break \symbol{95}start}|
|Breakpoint 1 at 0x400080: file envandargs.asm, line 6.|
|(gdb) \textbf{run}|
|Starting program: /home/jmates/envandargs a bb ccc|

|Breakpoint 1, \symbol{95}start () at envandargs.asm:6|
|6 \ \ \ \ \ \ \symbol{95}start: mov rbp,rsp \ \ \ \ ; save pointer to stack pointer|
|1: /x \symbol{36}rcx = 0x0|
|(gdb) \textbf{next}|
|7 \ \ \ \ \ \ \ \ \ \ \ \ \ \ pop rcx \ \ \ \ \ \ \ \ ; get argument count off stack|
|1: /x \symbol{36}rcx = 0x0|
|(gdb) \textbf{next}|
|\symbol{95}nexta () at envandargs.asm:9|
|9 \ \ \ \ \ \ \symbol{95}nexta: pop rcx \ \ \ \ \ \ \ \ ; loop popping off arg pointers|
|1: /x \symbol{36}rcx = 0x4|
|(gdb)|
\end{lstlisting}

One confusion with \texttt{gdb} is that it shows the line of code it is
about to run, not what it just executed. \\

The \texttt{pop rcx} has placed \texttt{4} into the \texttt{RCX}
register; this is the argument count, or what would be available in
\texttt{int main (int argc, ...} in a C program. The argument count
includes the program name plus here the arguments \texttt{a},
\texttt{bb}, and \texttt{ccc}.

\begin{lstlisting}
|(gdb) \textbf{next}|
|\symbol{95}nexta () at envandargs.asm:10|
|10 \ \ \ \ \ \ \ \ \ \ \ \ \ cmp rcx,0 \ \ \ \ \ \ ; ... until 64\symbol{45}bit 0 value|
|1: /x \symbol{36}rcx = 0x7fffffffeef2|
|(gdb)|
\end{lstlisting}

\texttt{RCX} should now contain the address of the first argument; let's
inspect that, looking for a \texttt{s} string value--a list of
not-\texttt{nul} characters followed by a \texttt{nul} character.
Actually let's cheat and peek at several string values\ldots

\begin{lstlisting}
|(gdb) \textbf{x /s 0x7fffffffeef2}|
|0x7fffffffeef2: "/home/jmates/envandargs"|
|(gdb) \textbf{x /5s 0x7fffffffeef2}|
|0x7fffffffeef2: "/home/jmates/envandargs"|
|0x7fffffffef0a: "a"|
|0x7fffffffef0c: "bb"|
|0x7fffffffef0f: "ccc"|
|0x7fffffffef13: "USER=jmates"|
|(gdb)|
\end{lstlisting}

These are the arguments directly followed by the environment variables.
Let's step back and take a look at the stack. The original stack pointer
was saved into the \texttt{RBP} register as the very first instruction.
Since this is a 64-bit system, we'll look at ten of what \texttt{gdb}
calls \texttt{g}iant or eight byte chunks:

\begin{lstlisting}
|(gdb) \textbf{x /10xg \symbol{36}rbp}|
|0x7fffffffed20: 0x0000000000000004 0x00007fffffffeef2|
|0x7fffffffed30: 0x00007fffffffef0a 0x00007fffffffef0c|
|0x7fffffffed40: 0x00007fffffffef0f 0x0000000000000000|
|0x7fffffffed50: 0x00007fffffffef13 0x00007fffffffef1f|
|0x7fffffffed60: 0x00007fffffffef2a 0x00007fffffffef92|
|(gdb)|
\end{lstlisting}

There's the argument count \texttt{4} followed by the four addresses
that point to the values of those four arguments, then the zero value
that divides the arguments from the environment--the "nothing" I spoke
of earlier--and then some of the addresses of the environment variables.
\texttt{gdb} can also shell out to hex viewers; this is another way to
show the contents of the arguments:

\begin{lstlisting}
|(gdb) \textbf{dump binary memory ddd 0x00007fffffffeef2 0x00007fffffffef1f}|
|(gdb) \textbf{shell xxd ddd}|
|...|
|(gdb) \textbf{quit}|
\end{lstlisting}

All very fascinating. Though, how can we add a duplicate environment
variable into the environment list?

\subsection*{High Level Language}

C, a high level language,\footnote{This term is relative.} uses the
exact same stack of values as the assembly only with handy names
attached to these addresses, plus a bunch of additional
cruft.\footnote{Ease and additional cruft are hallmarks of high level
languages.} Here is the file \texttt{envdup.c}.

\lstinputlisting{code/envdup.c}

This is similar to the shell script

\begin{lstlisting}
|\symbol{35}!/bin/sh|
|FOO=one FOO=two exec "\symbol{36}@"|
\end{lstlisting}

only that the shell does not let you create duplicate environment
variables.

\begin{lstlisting}
$ |\textbf{FOO=one FOO=two env \symbol{124} grep FOO}|
|FOO=two|
$ |\textbf{CFLAGS=\symbol{45}g make envdup}|
|cc \symbol{45}g envdup.c \symbol{45}o envdup|
$ |\textbf{gdb ./envdup}|
|...|
\end{lstlisting}

Let us first poke around in the debugger and inspect what C calls
\texttt{argv}.

\begin{lstlisting}
|(gdb) \textbf{set args a bb ccc}|
|(gdb) \textbf{break main}|
|Breakpoint 1 at 0x100000db6: file envdup.c, line 10.|
|(gdb) \textbf{run}|
|...|
|Breakpoint 1, main (argc=4, argv=0x7fff5fbff648) at envdup.c:10|
|10 \ \ \ \ \ \ \ \ \ \ \ int envcount = 0;|
|(gdb) \textbf{x /10xg 0x7fff5fbff648}|
|0x7fff5fbff648: 0x00007fff5fbff840 0x00007fff5fbff859|
|0x7fff5fbff658: 0x00007fff5fbff85b 0x00007fff5fbff85e|
|0x7fff5fbff668: 0x0000000000000000 0x00007fff5fbff862|
|0x7fff5fbff678: 0x00007fff5fbff87d 0x00007fff5fbff888|
|0x7fff5fbff688: 0x00007fff5fbff891 0x00007fff5fbff8cc|
|(gdb) \textbf{x /5s 0x00007fff5fbff840}|
|0x7fff5fbff840: "/Users/jmates/tmp/envdup"|
|0x7fff5fbff859: "a"|
|0x7fff5fbff85b: "bb"|
|0x7fff5fbff85e: "ccc"|
|0x7fff5fbff862: "\symbol{95}=/Users/jmates/tmp/envdup"|
|(gdb) \textbf{quit}|
|...|
\end{lstlisting}

Exactly like the assembly code, and even the same format on both
Linux assembly and here C code being run on Mac OS X. Now to run the
program proper:

\begin{lstlisting}
|\symbol{36} \textbf{unset FOO}|
|\symbol{36} \textbf{./envdup env \symbol{124} grep FOO}|
|FOO=one|
|FOO=two|
\end{lstlisting}

Duplicate environment entries for \texttt{FOO} have been set, as
reported by \texttt{env(1)}. \texttt{envdup} can be used to test what
other languages or programs do with duplicate environment entries. For
instance, is the first or second duplicate selected?

\begin{lstlisting}
|\symbol{36} \textbf{unset FOO}|
|\symbol{36} \textbf{./envdup bash \symbol{45}c 'echo \symbol{36}FOO; exec env' \symbol{124} egrep 'one\symbol{124}two'}|
|two|
|FOO=two|
|\symbol{36} \textbf{./envdup zsh \symbol{45}c 'echo \symbol{36}FOO; exec env' \symbol{124} egrep 'one\symbol{124}two'}|
|one|
|FOO=one|
\end{lstlisting}

\texttt{bash} selects the last, and \texttt{zsh} the first, and the
duplicate has been removed. What about other languages?

\begin{lstlisting}
|\symbol{36} \textbf{./envdup expect \symbol{45}c 'puts \symbol{36}env(FOO);puts \symbol{91}exec env\symbol{93}' \symbol{124} egrep 'one\symbol{124}two'}|
|one|
|FOO=one|
|FOO=two|
|\symbol{36} \textbf{./envdup perl \symbol{45}E 'say \symbol{36}ENV\symbol{123}FOO\symbol{125};exec "env"' \symbol{124} egrep 'one\symbol{124}two'}|
|one|
|FOO=one|
|\symbol{36} \textbf{./envdup ruby \symbol{45}e 'puts ENV\symbol{91}"FOO"\symbol{93};STDOUT.flush;exec "env"' \symbol{124} egrep 'one\symbol{124}two'}|
|one|
|FOO=one|
|FOO=two|
\end{lstlisting}

Consistent selection of the first duplicate, except that Ruby and TCL
have not cleaned up the environment and pass the duplicates on to
\texttt{env}. Perl instead matches \texttt{zsh} and calls \texttt{env}
with the duplicate removed.

\begin{lstlisting}
|\symbol{36} \textbf{expect \symbol{45}v}|
|expect version 5.45|
|\symbol{36} \textbf{perl \symbol{45}v \symbol{124} sed \symbol{45}n 2p}|
|This is perl 5, version 32, subversion 1 (v5.32.1) built for amd64\symbol{45}openbsd|
|\symbol{36} \textbf{ruby \symbol{45}v}|
|ruby 3.0.1p64 (2021\symbol{45}04\symbol{45}05 revision 0fb782ee38) \symbol{91}x86\symbol{95}64\symbol{45}openbsd\symbol{93}|
\end{lstlisting}

Duplication of environment variables is not a trivial issue; see
\url{https://www.sudo.ws/repos/sudo/rev/d4dfb05db5d7} for the security
patch to \texttt{sudo(8)}. In particular, if only the first environment
variable is sanitized \texttt{bash} as shown above will use the second
variable. This could allow an attacker to specify a custom second
\texttt{PATH} or various \texttt{LD\symbol{95}*} environment values that
a subsequent \texttt{bash} process would use. ``Practical UNIX \&
Internet Security''\cite{simson1996} mentions the duplicate environment
variable attack;

\begin{quote}
``examine the environment to be certain that there is only one instance
of the variable: the one you set. An attacker can run your code from
another program that creates multiple instances of an environment
variable. Without an explicit check, you may find the first instance,
but not the others; such a situation could result in problems later
on.'' p.717
\end{quote}

\texttt{sudo} was patched for this two decades later. Other software
remains completely unpatched; one idea is that perhaps libc should
maybe somehow handle the duplicates before the program enters
\texttt{main}? \\

\url{https://sourceware.org/bugzilla/show_bug.cgi?id=19749} \\

This may be difficult given the differences between \texttt{bash} and
other software with regard to which duplicate is passed through\ldots
what does your language do?

\section*{\texttt{execv(3)} versus \texttt{system(3)}}

These calls are used to execute new programs. Understanding the
difference between \texttt{exec(3)} and \texttt{system(3)} is important
as \texttt{system(3)} uses the shell while passing a shell command to
\texttt{exec(3)} may not fare so well. The difference is complicated by
languages that call \texttt{system(3)} \texttt{exec}--TCL, as noted
above. Also, PHP. Some software make both interfaces available.
\texttt{git(1)} in particular offers both \texttt{GIT\symbol{95}SSH} and
\texttt{GIT\symbol{95}SSH\symbol{95}COMMAND} environment variables:

\begin{lstlisting}
| \ \ \ \ \ \ \ \symbol{36}GIT\symbol{95}SSH\symbol{95}COMMAND takes precedence over \symbol{36}GIT\symbol{95}SSH, and is interpreted|
| \ \ \ \ \ \ \ by the shell, which allows additional arguments to be included.|
| \ \ \ \ \ \ \ \symbol{36}GIT\symbol{95}SSH on the other hand must be just the path to a program (which|
| \ \ \ \ \ \ \ can be a wrapper shell script, if additional arguments are needed).|
\end{lstlisting}

Less well documented programs may need a source code dive or
\texttt{strace} to determine which is used. A shell command ``not
working'' is a good indication that \texttt{exec(3)} is being used
instead of \texttt{system(3)}. A minimal \texttt{exec(3)} example would
be the following C code,

\lstinputlisting{code/minexec.c}

which can illustrate the ``\texttt{system} was assumed but \texttt{exec}
used'' problem:

\begin{lstlisting}
|\symbol{36} \textbf{make minexec}|
|gcc \ \ \ \ minexec.c \ \ \symbol{45}o minexec|
|\symbol{36} \textbf{./minexec echo 'hi there'}|
|hi there|
|\symbol{36} \textbf{./minexec 'echo hi there'}|
|minexec: exec failed: No such file or directory|
\end{lstlisting}

\texttt{strace} is most useful when a program omits details necessary to
debug an issue; what is actually being run for the second command is the
string \texttt{echo hi there} which typically does not exist in
\texttt{PATH}. No shell word split is done by \texttt{execv(3)}; the
string given as the first argument is searched for in \texttt{PATH}.
\texttt{system(3)} by contrast will pass a string to \texttt{sh}. For a
minimal C example the list of arguments must be converted into a single
string, ideally with more error checking and fewer assumptions than are
made here--this is something like \texttt{tr '\symbol{92}0' ' '} for the
arguments list.

\lstinputlisting{code/minsys.c}

This code by contrast does run \texttt{'echo hi there'} without error

\begin{lstlisting}
|\symbol{36} \textbf{make minsys}|
|cc \symbol{45}O2 \symbol{45}pipe \ \ \ \symbol{45}o minsys minsys.c|
|\symbol{36} \textbf{./minsys 'echo hi here'}|
|hi here|
\end{lstlisting}

as what is being run is \texttt{/bin/sh \symbol{45}c 'echo hi
there'}--compare the programs with \texttt{strace}. Redirections and
other shell constructs are likewise acceptable to \texttt{system}, but
not \texttt{exec}.

\section*{Maximum Arguments}

There exist various limits on what a unix program will accept. These
limits have been raised over time. The usual encounter will run
something like

\begin{lstlisting}
|\symbol{36} \textbf{ls *}|
\end{lstlisting}

in a directory with a very large number of files, or files with very
long names, or where many environment variables have been set; the shell
first expands the glob, and then calls some \texttt{exec} function with
the complete list of filenames. A typical limit is
\texttt{ARG\symbol{95}MAX} that includes both the arguments and the
environment list.

\begin{lstlisting}
|\symbol{36} \textbf{getconf ARG\symbol{95}MAX}|
|2097152|
|\symbol{36} \textbf{mkdir big \symbol{38}\symbol{38} cd big}|
|\symbol{36} \textbf{perl \symbol{45}e '\symbol{36}n = "a"x128; \symbol{36}c = 16384;' \symbol{92}}|
| \ \textbf{\symbol{45}e 'while (\symbol{36}c\symbol{45}\symbol{45}) \symbol{123} `touch \symbol{36}n`; \symbol{36}n++ \symbol{125}'}|
|\symbol{36} \textbf{ls *}|
|\symbol{45}bash: /bin/ls: Argument list too long|
\end{lstlisting}

Hence \texttt{xargs(1)}; this converts arbitrary amounts of standard
input into as many as necessary executions of a command with the
standard input converted to not more than the maximum allowed arguments.

\begin{lstlisting}
|\symbol{36} \textbf{ls \symbol{124} xargs echo \symbol{124} wc \symbol{45}l}|
|17|
\end{lstlisting}

Seventeen \texttt{echo} executions were necessary to get through the
list. Note that parsing \texttt{ls} is typically a bad idea; filenames
really should be passed around \texttt{nul}-delimited, or handled
internally by a process. Modern versions of \texttt{find} support either
the \texttt{\symbol{45}exec ... \symbol{123}\symbol{125} +} form to act
like \texttt{xargs} without the problems of \texttt{xargs}, or to simply
remove all these files there is the more efficient
\texttt{\symbol{45}delete} flag.

\begin{lstlisting}
|find . \symbol{45}mindepth 1 \symbol{45}exec rm \symbol{123}\symbol{125} +|
|find . \symbol{45}mindepth 1 \symbol{45}delete|
\end{lstlisting}

\texttt{ARG\symbol{95}MAX} is not be the only limit; another one
involves interpreted scripts, those with a ``hash bang'' or ``shebang''
line. \texttt{zsh} 5.3.1 defaulted to a static \texttt{POUNDBANGLIMIT}
of 64 characters in \texttt{Src/exec.c}. For a typical shebang line of

\begin{lstlisting}
|\symbol{35}!/usr/bin/env expect|
|puts "hello world"|
\end{lstlisting}

this is not a problem, though packages in NixOS are particularly adept
at running past this limit.

\begin{lstlisting}
|\symbol{35} \textbf{find /nix/store \symbol{45}type f \symbol{45}exec perl \symbol{45}nE \symbol{92}}|
| \ \textbf{'say length if /\symbol{94}\symbol{35}!/ and length \symbol{62} 64; close ARGV' \symbol{123}\symbol{125} +}|
|68|
|66|
|66|
|...|
|\symbol{35} \textbf{find /nix/store \symbol{45}type f \symbol{45}exec perl \symbol{45}nE \symbol{92}}|
| \ \textbf{'say length if /\symbol{94}\symbol{35}!/ and length \symbol{62} 64; close ARGV' \symbol{123}\symbol{125} + \symbol{92}}|
| \ \textbf{\symbol{124} sort \symbol{45}nr \symbol{124} head \symbol{45}1}|
|1466|
|\symbol{35}|
\end{lstlisting}

A solution? Increase the size of that buffer in \texttt{zsh}. I probably
should talk more about buffers, but that's another article\ldots

\bibliography{references}
\end{document}
