\documentclass[10pt,a4paper]{article}
\setlength{\topmargin}{-5mm}
\setlength{\textheight}{244mm}
\setlength{\oddsidemargin}{0mm}
\setlength{\textwidth}{165mm}

\setlength{\parindent}{0pt}

\usepackage{fancyvrb}
\DefineVerbatimEnvironment{verbatim}{Verbatim}{xleftmargin=4mm}

\usepackage{lmodern}
\usepackage[T1]{fontenc}
\usepackage{pxfonts}
\usepackage{textcomp}

\usepackage[square]{natbib}
\citestyle{nature}

\usepackage{listings}
\lstset{basicstyle=\ttfamily,
escapeinside={||}}

\frenchspacing

\title{The Shell Game}
\author{Jeremy Mates}
\date{June 27, 2021}

\usepackage{hyperref}
\hypersetup{pdfauthor={Jeremy Mates},pdftitle={The Shell Game}}

\begin{document}
\renewcommand{\thefootnote}{\alph{footnote}}
\bibliographystyle{plainnat}
\maketitle

While the unix shell is much used it may not be the best choice for
programming; some consider it little more than a useful means to
orchestrate commands and prototype things. Others have written extensive
software with it. I'm more in the former camp.

\begin{lstlisting}
|\symbol{36} \textbf{(echo one; echo -n two) \symbol{124} while read l; do echo "\symbol{36}l"; done}|
|one|
|\symbol{36} \textbf{(echo pa;echo re;echo -n ci) \symbol{124} while read l;do echo "\symbol{36}l";done}|
|pa|
|re|
|\symbol{36} |
\end{lstlisting}

Where did the last line go? Choices here are to double down on this code
(and maybe miss the occasional line?) or to detect and handle this
condition (longer, slower, and more complicated) or to use a different
language (my usual solution). However, the shell is common and may
intrude when debugging other problems so some knowledge of the shell and
its quirks is essential. Let's review some syntax.

\begin{lstlisting}
|\symbol{36} \textbf{echo a b c}|
|a b c|
|\symbol{36} \textbf{env letters="a b c" sh -c 'echo "\symbol{36}letters"'}|
|a b c|
\end{lstlisting}

Commands take arguments--in two forms. In addition to the usually
visible command line arguments there is a list of usually invisible
environment variables passed down by default to every child process.
Complications include code that runs before the command is run

\begin{lstlisting}
|\symbol{36} \textbf{echo a >b c}|
|\symbol{36} \textbf{cat b}|
|a c|
|\symbol{36} \textbf{rm b}|
\end{lstlisting}

and various order of operation issues:

\begin{lstlisting}
|\symbol{36} \textbf{unset FOO}|
|\symbol{36} \textbf{FOO=bar echo ">\symbol{36}FOO<"}|
|>\symbol{60}|
\end{lstlisting}

In the first example standard output is redirected into the file
\texttt{b} and the remaining arguments \texttt{a c} are passed to
\texttt{echo}. In the second case \texttt{\symbol{36}FOO} is first
interpolated to nothing because \texttt{FOO} is at that point unset, and
then the \texttt{FOO=bar} assignment happens, and then \texttt{echo}
runs without any arguments because \texttt{\symbol{36}FOO} was already
interpolated away. This is akin to breaking eggs over a stove and then
putting the pan on the stove and then wondering why there's a mess of
eggs under the pan. \\

\texttt{set -x} is useful:

\begin{lstlisting}
|\symbol{36} \textbf{unset FOO}|
|\symbol{36} \textbf{set -x}|
|\symbol{36} \textbf{FOO=bar echo ">\symbol{36}FOO<"}|
|+ echo '><'|
|+ FOO=bar|
|><|
|\symbol{36} \textbf{set +x}|
|+ set +x|
\end{lstlisting}

This shows that the \texttt{echo} indeed ran without argument. It will
not tell someone unfamiliar with the shell why \texttt{\symbol{36}FOO}
was already interpolated away; debugging must be paired with at least
some understanding of how the software works. Still, the trace output
should help as it may better show what is going on, especially as the
problems become more complicated and less clear or when requesting help
from others. \\

Such order of operation issues are by no means exclusive to the shell;
ECMAScript 6 has a ``Temporal Dead Zone'' where bindings are not usable
before their declaration, or what could be called the ``eggs on the
stove before pan'' pattern.

\begin{lstlisting}
|\symbol{36} \textbf{node -e 'console.log(typeof x); let x = 42'}|
|...|
|ReferenceError: Cannot access 'x' before initialization|
\end{lstlisting}

Many other examples can doubtless be found. One solution in the shell is
to separate the assignment from the usage (the ``avoid doing too many
things on one line'' pattern). This can take a variety of forms:

\begin{lstlisting}
|\symbol{36} \textbf{unset FOO}|
|\symbol{36} \textbf{FOO=bar}|
|\symbol{36} \textbf{printf "\symbol{37}s\symbol{92}n\symbol{92}n" \symbol{36}FOO}|
|bar|

|\symbol{36} \textbf{unset FOO}|
|\symbol{36} \textbf{FOO=bar ; printf "\symbol{37}s\symbol{92}n\symbol{92}n" \symbol{36}FOO}|
|bar|

|\symbol{36} \textbf{unset FOO}|
|\symbol{36} \textbf{FOO=bar sh -c 'echo \symbol{36}FOO'}|
|bar|
\end{lstlisting}

These adjustments assume familiarity with the shell; \texttt{FOO=bar
echo \symbol{36}FOO} produces no error as it is not invalid, much as one
can construct a grammatically correct phrase that is flawed in some
other way. A perhaps useful set of terms from spoken languages: the
vocabulary, the grammar, and the usage. For the shell the vocabulary
would in part consist of

\begin{lstlisting}
|cd \ \ \ \ \ \ \ \ \ while \ \ \ \ \ \ \ \ ls \ \ echo \ \ \symbol{38} \ \ \ \ \ somevar \ \ \ SECRET\symbol{95}API\symbol{95}KEY|
|printf \ \ \ \ \ \symbol{60} \ \ \ \ \ \ \ \ \ \ \ \ \symbol{45}\symbol{45} \ \ \symbol{36} \ \ \ \ \ read \ \ \symbol{62} \ \ \ \ \ \ \ \ \ =|
|/dev/null \ \ /etc/passwd \ \ do \ \ l \ \ \ \ \ \symbol{45}rf \ \ \ done \ \ \ \ \ \ rm|
\end{lstlisting}

and the grammar how the vocabulary can be strung together

\begin{lstlisting}
|\symbol{36} \textbf{>/dev/null ls}|
|\symbol{36} \textbf{>\symbol{38}2 echo \symbol{36}somevar}|
\end{lstlisting}

while the usage indicates which grammatical forms are most suitable.
This is the domain of style guides and best practices--or knowing when
not to use the shell. Redirections are more commonly used after rather
than before the command even though they may be legal in other
locations:

\begin{lstlisting}
|\symbol{36} \textbf{>/dev/null ls} \ \ \ \ \ \ \ \ \symbol{35} strange|
|\symbol{36} \textbf{ls >/dev/null} \ \ \ \ \ \ \ \ \symbol{35} more typical|

|\symbol{36} \textbf{</etc/passwd while read l; do echo \symbol{36}l; done}|
|\textit{?}|
|\symbol{36} \textbf{printf >\symbol{38}2 \symbol{45}\symbol{45} "\symbol{36}somevar\symbol{92}n"}|
\end{lstlisting}

The \texttt{</etc/passwd} redirection before \texttt{while} is
legal in \texttt{zsh}--a usage choice--and illegal in other shells--a
grammar error:

\begin{lstlisting}
|\symbol{36} \textbf{</etc/passwd while read l; do echo \symbol{36}l; done}|
|/opt/local/bin/mksh: syntax error: unexpected 'do'|
|\symbol{36} \textbf{exec zsh}|
|\symbol{37} \textbf{</etc/passwd while read l; do echo \symbol{36}l; done}|
|...|
\end{lstlisting}

The conventional form should be used in conventional scripts. While
debugging ``anything goes'' though one may want to clean things up
before presenting a problem elsewhere. Or at least to caveat the
code as such. Non-conventional uses may help reveal bugs or
portability problems. \\

Let us study the ``\texttt{while} loop drops data'' issue some more.
Where did the last line go?

\begin{lstlisting}
|\symbol{36} \textbf{(echo one;echo -n two)\symbol{124}while read l;do echo ">\symbol{36}l<";done}|
|>one<|
|\symbol{36} \textbf{printf ",\symbol{37}s,\symbol{92}n" "\symbol{36}l"}|
|\symbol{44}\symbol{44}|
\end{lstlisting}

It did not escape into \texttt{\symbol{36}l} following the loop, and
presumably the \texttt{do ...} expression was only entered once, which
the \texttt{><} or \texttt{\symbol{44}\symbol{44}}
decorators\footnote{Such ``decorators'' are useful when there are
invisible characters not otherwise seen--various whitespace and control
characters in ASCII, or non-breaking spaces many other problematic
characters in Unicode and other encodings. A hex dumper may also help.
On decorators, use whatever you want, though beware decorators that can
be mistaken for valid outputs.} help establish. Therefore the
\texttt{while} or the \texttt{read} or a combination of the two is
likely to blame. How can these be tested in isolation? (Another method
would be to read the \texttt{read} documentation.)

\begin{lstlisting}
|\symbol{36} \textbf{cat input}|
|one|
|two|
|\symbol{36} \textbf{cat code}|
|read l; echo "1:\symbol{36}l:"|
|read l; echo "2:\symbol{36}l:"|
|read l; echo "3:\symbol{36}l:"|
|\symbol{36} \textbf{chmod +x code}|
|\symbol{36} \textbf{./code < input}|
|1:one:|
|2:two:|
|3::|
\end{lstlisting}

The three \texttt{read} calls ran despite there being only two lines
of input. This points to \texttt{while} failing, and if we're
relatively awake (which is, alas, not always possible while debugging)
the logical answer is because \texttt{read} returned a false value. We
can test this hypothesis.

\begin{lstlisting}
|\symbol{36} \textbf{cat code}|
|read l; echo ":\symbol{36}l:\symbol{36}?"|
|read l; echo ":\symbol{36}l:\symbol{36}?"|
|read l; echo ":\symbol{36}l:\symbol{36}?"|
|\symbol{36} \textbf{./code < input}|
|:one:0|
|:two:0|
|::1|
\end{lstlisting}

What if when \texttt{read} returns non-zero (false) we run something else?

\begin{lstlisting}
|\symbol{36} \textbf{(echo one;echo \symbol{45}n two)\symbol{124}while read l \symbol{124}\symbol{124} echo \symbol{36}l ;do echo \symbol{36}l;done}|
|one|
|two|
|two|
\end{lstlisting}
\clearpage

\begin{lstlisting}
|\symbol{94}C\symbol{36} \symbol{94}C|
|\symbol{36} \textbf{\symbol{94}C}|
|\symbol{36}|
\end{lstlisting}

Now there are two instances of \texttt{two} printed, and we had to
recover the terminal, here by mashing \textbf{control+c} a lot. Also
note how without a decorator it may not be clear which instance of
\texttt{echo} is printing what. Another good idea is to slow things
down. Lacking a ``turbo button'' to press, we add a \texttt{sleep}
statement.

\begin{lstlisting}
|\symbol{36} \textbf{(echo 1;echo \symbol{45}n 2)\symbol{124}while read l \symbol{124}\symbol{124} echo "A\symbol{36}l"; do echo "B\symbol{36}l"; sleep 1; done}|
|B1|
|A2|
|B2|
|A|
|B|
|\symbol{94}C\symbol{36}|
\end{lstlisting}

With the \texttt{|| echo ...} code turned into a logic test, things are
looking better:

\begin{lstlisting}
$ |\textbf{(echo one;echo \symbol{45}n two)\symbol{124}while read l \symbol{124}\symbol{124} [ \symbol{45}n "\symbol{36}l" ];do echo "\symbol{36}l"; done}|
one
two
\end{lstlisting}

There are still problems with this; one may want to set \texttt{IFS} and
to use the \texttt{-r} flag to \texttt{read}. And to ensure the the
variable is unset prior to the loop. Also, \texttt{less(1)} may have a
perhaps interesting \texttt{-p} flag.

\begin{lstlisting}
|\symbol{36} \textbf{man ksh \symbol{124} less \symbol{45}p '\symbol{94} *read '}|
|...|
\end{lstlisting}

\section*{Concerning Globs}

\begin{lstlisting}
|\symbol{36} \textbf{mkdir testdir \symbol{38}\symbol{38} cd testdir}|
|\symbol{36} \textbf{touch \symbol{45}\symbol{45} \symbol{45}rf}|
|\symbol{36} \textbf{ls *f}|
|.   ..   -rf|
\end{lstlisting}

why were \texttt{.} and \texttt{..} displayed? Use \texttt{set \symbol{45}x} to trace the shell.

\begin{lstlisting}
|\symbol{36} \textbf{set \symbol{45}x}|
|\symbol{36} \textbf{ls *f}|
|+ ls -rf|
|.   ..   -rf|
\end{lstlisting}

This shows that \texttt{ls \symbol{45}rf} is run. What do those flags do?

\begin{lstlisting}
|\symbol{36} \textbf{man ls \symbol{124} col \symbol{45}bx \symbol{124} egrep '\symbol{94} *\symbol{45}[rf]'}|
| \ \ \ \ \symbol{45}f \ \ \ \ \ Output is not sorted. \ This option implies \symbol{45}a.|
| \ \ \ \ \symbol{45}r \ \ \ \ \ Reverse the order of the sort to get reverse lexicographical|
|\symbol{36} \textbf{man ls \symbol{124} col \symbol{45}bx \symbol{124} egrep '\symbol{94} *\symbol{45}a'}|
| \ \ \ \ \symbol{45}a \ \ \ \ \ Include directory entries whose names begin with a dot (`.').|
\end{lstlisting}

The \texttt{*f} glob has matched the file \texttt{\symbol{45}rf} which
the shell then passes to \texttt{ls} which in turn parses
\texttt{\symbol{45}rf} as a bundle of command line flags. This has
various security implications. Option processing should ideally be
disabled prior to a glob, which on most systems and for most commands
these days is done by supplying \texttt{\symbol{45}\symbol{45}} after
the last flag and before any external, untrusted input. See
\texttt{getopt(3)} for details.

\begin{lstlisting}
|\symbol{36} \textbf{( set \symbol{45}x; ls \symbol{45}\symbol{45} *f )}|
|+ ls \symbol{45}- -rf|
|-rf|
|\symbol{36} \textbf{rm \symbol{45}\symbol{45} \symbol{45}rf}|
|\symbol{36} \textbf{ls}|
|\symbol{36} \textbf{cd ..}|
|\symbol{36} \textbf{rmdir testdir}|
\end{lstlisting}

Globs are handled differently in different shells; in an empty directory
(but note that \texttt{/var/empty}\footnote{This footnote was
intentionally left blank.} is not actually empty on all platforms, or
may not exist):

\begin{lstlisting}
|\symbol{36} \textbf{cd /var/empty}|
|\symbol{36} \textbf{echo *}|
|*|
|\symbol{36} \textbf{PS1='\symbol{37}\symbol{37} ' zsh \symbol{45}f}|
|\symbol{37} \textbf{echo *}|
|zsh: no matches found: *|
\end{lstlisting}
 
I consider bare globs that are assumed to match nothing unsafe as they
are a file creation or working directory change away from suddenly
matching something and then who knows what goes wrong when a filename or
command flag appears instead of the literal glob. This is of less
concern during interactive use--``whoa, that was strange''--and of great
concern when some forgotten shell workflow is blowing up and nobody
knows why. \\

\texttt{find(1)} also accepts globs; the result may vary depending on
whether the shell or \texttt{find} performs the glob expansion.

\begin{lstlisting}
|\symbol{37} \textbf{mkdir testdir \symbol{38}\symbol{38} cd testdir}|
|\symbol{37} \textbf{mkdir \symbol{45}p \symbol{123}a,b\symbol{125}/\symbol{123}xa,xb\symbol{125}}|
|\symbol{37} \textbf{find . \symbol{45}name x*}|
|zsh: no matches found: x*|
|\symbol{37} \textbf{exec mksh}|
|\symbol{36} \textbf{find . \symbol{45}name x*}|
|./a/xa|
|./a/xb|
|./b/xa|
|./b/xb|
|\symbol{36} \textbf{touch xom}|
|\symbol{36} \textbf{find . \symbol{45}name x*}|
|./xom|
|\symbol{36} \textbf{find . \symbol{45}name "x*"}|
|./a/xa|
|./a/xb|
|./b/xa|
|./b/xb|
|./xom|
\end{lstlisting}

A useful use of a shell glob would be to provide a list of input
directories (and, maybe, files) to \texttt{find}:

\begin{lstlisting}
|\symbol{36} \textbf{find \symbol{45}\symbol{45} * \symbol{45}name "xa"}|
|a/xa|
|b/xa|
|\symbol{36} \textbf{echo find \symbol{45}\symbol{45} * \symbol{45}name "xa"}|
|find \symbol{45}- a b xom -name xa|
\end{lstlisting}

Another trick is the \texttt{zsh \symbol{45}f} used above. This disables
any user-supplied configuration; if a problem persists then it is likely
a \texttt{zsh} issue; otherwise, a next step would be to bisect the
\texttt{zsh} configuration to find where the issue is coming from.
Similar flags exist for other software, e.g. \texttt{vim \symbol{45}u
NONE}. This pattern can be abstracted and used to ask whether an issue
is, say, at the application or network level, assuming a suitable test
exists that would distinguish an application issue from a network one:
is the error present on the wire, or not? If successfully applied this
should help direct attention to the component where the problem is.
\texttt{zsh \symbol{45}f} may be too minimal though from such a shell
one can apply only what is necessary to test the problem.

\section*{POSIX Word Split}

POSIX word split is another shell convention that must be guarded against.

\begin{lstlisting}
|\symbol{36} \textbf{touch "a file"}|
|\symbol{36} \textbf{var="a file"}|
|\symbol{36} \textbf{echo \symbol{36}var}|
|a file|
\end{lstlisting}

Thanks to POSIX word split \texttt{a file} is actually split into two distinct
arguments that \texttt{echo} hides and \texttt{rm} reveals.

\begin{lstlisting}
|\symbol{36} \textbf{rm \symbol{45}rf \symbol{36}var}|
|rm: a: No such file or directory|
|rm: file: No such file or directory|
\end{lstlisting}

When \texttt{\symbol{36}var} contains something like \texttt{/usr/
somedir} this has on many occasions resulted in the \texttt{/usr} or
\texttt{/var} or whatever directory tree being completely destroyed.
\texttt{zsh} does not implement POSIX word split (by default) and passes
\texttt{a file} as a single argument to \texttt{rm}.

\begin{lstlisting}
|\symbol{36} \textbf{exec zsh}|
|\symbol{37} \textbf{var="a file"}|
|\symbol{37} \textbf{print \symbol{45}l \symbol{36}var}|
a file
|\symbol{37} \textbf{( setopt SH\symbol{95}WORD\symbol{95}SPLIT ; print \symbol{45}l \symbol{36}var )}|
a
file
|\symbol{37} \textbf{rm \symbol{36}var}|
|\symbol{37} \textbf{ls}|
|\symbol{37} |
\end{lstlisting}

\subsection*{POSIX Word Glob}

But wait, there's more! In addition to the POSIX word split, a glob is
also performed.

\begin{lstlisting}
|\symbol{36} \textbf{touch c.txt a.txt}|
|\symbol{36} \textbf{sleep 3; touch b.txt}|
|\symbol{36} \textbf{var="*.txt \symbol{91}abc\symbol{93}*"}|
|\symbol{36} \textbf{set \symbol{45}x}|
|\symbol{36} \textbf{ls \symbol{36}var}|
|+ ls a.txt b.txt c.txt a.txt b.txt c.txt|
|a.txt       a.txt   b.txt   b.txt   c.txt   c.txt|
\end{lstlisting}

Here, the words \texttt{*.txt} and \texttt{\symbol{91}abc\symbol{93}*}
are first split, and then each is globbed. Note also how \texttt{ls}
sorts the results; this may give the false impression that the glob or
filesystem impose a specific ordering that is not actually present. \\

The split can be disabled by quoting the variable:

\begin{lstlisting}
|\symbol{36} \textbf{var="a file"}|
|+ var='a file'|
|\symbol{36} \textbf{echo "\symbol{36}var" "\symbol{36}var" "\symbol{36}var"}|
|+ echo 'a file' 'a file' 'a file'|
|a file a file a file|
\end{lstlisting}

quoting also disables the glob:

\begin{lstlisting}
|\symbol{36} \textbf{var="*.txt"}|
|+ var='*.txt'|
|\symbol{36} \textbf{echo "\symbol{36}var"}|
|+ echo '*.txt'|
|*.txt|
|\symbol{36} \textbf{rm \symbol{36}var}|
|+ rm a.txt b.txt c.txt|
\end{lstlisting}

In general POSIX shell variables must be quoted to avoid the often
problematic split (and glob!), unless one does want the split (and
glob!) behavior. In such a case I would probably insert a comment that
states why that variable must not be quoted to help reduce the chance of
it being incorrectly quoted by a future maintainer. \\

The lack of a split in \texttt{zsh} can cause problems should a split be
desired, though there is a flag to apply it:

\begin{lstlisting}
|\symbol{36} \textbf{zsh}|
|\symbol{37} \textbf{p='print \symbol{45}l'}|
|\symbol{37} \textbf{\symbol{36}p foo bar}|
|zsh: command not found: print -l|
|\symbol{37} \textbf{\symbol{36}=p foo bar}|
|foo|
|bar|
\end{lstlisting}

An important point here is that \texttt{print \symbol{45}l} could be a
command name; unix filenames are sequences of bytes that disallow only
two characters (\texttt{nul} and \texttt{/}). A \texttt{print
\symbol{45}l} command could be placed into a \texttt{PATH} directory
and executed:

\begin{lstlisting}
|\symbol{37} \textbf{mkdir badidea \symbol{38}\symbol{38} cd badidea}|
|\symbol{37} \textbf{path+=`pwd`}|
|\symbol{37} \textbf{print "echo not a good idea" > "print \symbol{45}l"}|
|\symbol{37} \textbf{chmod +x print\symbol{92} \symbol{45}l}|
|\symbol{37} \textbf{print\symbol{92} \symbol{45}l}|
|not a good idea|
|\symbol{37} \textbf{var="print \symbol{45}l"}|
|\symbol{37} \textbf{\symbol{36}var}|
|not a good idea|
|\symbol{37} |
\end{lstlisting}

But probably should not be. Unexpected characters in filenames mostly
cause hard to debug problems, especially where Windows (or Internet)
\texttt{\symbol{92}r\symbol{92}n} linefeeds have gotten into a script,
in which case the \texttt{\symbol{92}r} is considered part of a
filename, and not a linebreak:

\begin{lstlisting}
|\symbol{35} \textbf{bash \symbol{45}\symbol{45}version \symbol{124} sed \symbol{45}n 1p}|
|GNU bash, version 2.05a.0(1)-release (i686-pc-linux-gnu)|
|\symbol{35} \textbf{echo ls\symbol{36}'\symbol{92}r' > cmd}|
|\symbol{35} \textbf{chmod +x cmd ; ./cmd}|
|: command not found|
\end{lstlisting}

The error reporting for this condition has been improved in modern
shells, and in any case a hex viewer will reveal any invisible
characters:

\begin{lstlisting}
|\symbol{36} \textbf{printf '\symbol{35}!/usr/bin/env zsh\symbol{92}nls\symbol{92}r\symbol{92}n' > cmd}|
|\symbol{36} \textbf{chmod +x cmd ; ./cmd}|
|./cmd:2: command not found: ls\symbol{94}M|
|\symbol{36} \textbf{od \symbol{45}c cmd}|
|0000000 \ \ \ \symbol{35} \ \ ! \ \ / \ \ u \ \ s \ \ r \ \ / \ \ b \ \ i \ \ n \ \ / \ \ e \ \ n \ \ v \ \ \ \ \ \ z|
|0000020 \ \ \ s \ \ h \ \symbol{92}n \ \ l \ \ s \ \symbol{92}r \ \symbol{92}n|
|0000027|
\end{lstlisting}

One could create a \texttt{ls\symbol{94}M} command in the \texttt{PATH}
(perhaps via \textbf{control+v} \textbf{control+m} to produce that
\texttt{\symbol{94}M}), but a better idea would be to fix the script to
not contain inappropriate invisible characters.

\section*{Null Interpolation}

Consider a build directory cleanup script, which might contain

\begin{lstlisting}
|rm -rf build/*|
\end{lstlisting}

This is far too static--what if we need to vary the build directory?
Hence the improved form

\begin{lstlisting}
|rm \symbol{45}rf "\symbol{36}\symbol{123}build\symbol{125}/*"|
\end{lstlisting}

where care has been taken to properly quote \texttt{\symbol{36}build} to
prevent the POSIX word split (and glob!) thing from nuking inappropriate
directory trees. \\

The disaster requires two conditions. First, that the code is run as
\texttt{root}. Second, that someone forgets to set
\texttt{\symbol{36}build}. In this case the command that is run is

\begin{lstlisting}
|rm \symbol{45}rf /*|
\end{lstlisting}

This might be a good time to mention backups, snapshots, configuration
management, and various other ways to restore or rebuild a working
environment. The shell has long contained features that guard against
such null interpolation; the ``Heirloom Bourne Shell'' supports:

\begin{lstlisting}
| \ \ \ set \symbol{91}\symbol{45}\symbol{45}aefhknptuvx \symbol{91}arg ...\symbol{93}\symbol{93}|
| \ \ \ ...|
| \ \ \ \ \ \symbol{45}u \ Treat unset variables as an error when substituting.|
| \ \ \ ...|
| \ \ \ \symbol{36}\symbol{123}parameter:?word\symbol{125}|
| \ \ \ \ \ If parameter is set and not empty then substitute its value;|
| \ \ \ \ \ otherwise, print word and exit from the shell. ...|
\end{lstlisting}

or, manual checks that necessary variables are set can be made. The
following is fairly similar to the
\texttt{\symbol{36}\symbol{123}parameter:?word\symbol{125}} expansion.

\begin{lstlisting}
|\symbol{35}!/bin/sh|
|if \symbol{91} \symbol{45}z "\symbol{36}build" \symbol{93}; then|
| \ \ echo \symbol{62}\symbol{38}2 "\symbol{36}0: 'build' variable is not set"|
| \ \ exit 1|
|fi|
|...|
\end{lstlisting}

\section*{Subshells and Processes}

Earlier a subshell was used without naming it:

\begin{lstlisting}
|\symbol{36} \textbf{(set \symbol{45}x;ls \symbol{45}\symbol{45} *f)}|
\end{lstlisting}

This isolates the \texttt{set \symbol{45}x} change to a child process.
The isolation also applies to shell variables:

\begin{lstlisting}
|\symbol{36} \textbf{var=orig}|
|\symbol{36} \textbf{( var=new ; echo "\symbol{36}var" )}|
|new|
|\symbol{36} \textbf{echo "\symbol{36}var"}|
|orig|
\end{lstlisting}

This isolation can manifest in unexpected and perhaps undesirable ways,
for instance in a \texttt{while} loop:

\begin{lstlisting}
|\symbol{36} \textbf{var=orig}|
|\symbol{36} \textbf{echo new \symbol{124} while read l; do var=\symbol{36}l; echo \symbol{36}var; done}|
|new|
|\symbol{36} \textbf{echo "\symbol{36}var"}|
|orig|
\end{lstlisting}

Child processes cannot alter the parent process; this is why the
\texttt{orig} value in \texttt{var} reappears after the subshell
completes. Actually, the previous statement is slightly false; a child
process can change the parent because certain resources are shared
between them. Identification of subshells is possible with process
listing tools, or via methods specific and internal to the shell, or by
various system call or kernel tracing methods. (Or when your shell code
does something unexpected that reveals a subshell.) \\

Process listings are transitory and shell commands may complete very
quickly, especially when they fail. One trick is to add \texttt{sleep}
calls to slow down the speed of execution:

\begin{lstlisting}
|\symbol{37} \textbf{( sleep 31; echo new ) \symbol{92}}|
|\textbf{\symbol{124} while read l; do var=\symbol{36}l; echo \symbol{36}var; sleep 33; done}|
\end{lstlisting}

In another terminal inspect the process listing; \texttt{pstree(1)} is
useful for this:

\begin{lstlisting}
|\symbol{37} \textbf{pstree > x}|
|\symbol{37} \textbf{grep \symbol{45}2 sleep x}|
| \symbol{124} \symbol{124}\symbol{45}+= 00528 jmates zsh \symbol{45}l|
| \symbol{124} \symbol{124} \symbol{92}\symbol{45}+= 30192 jmates zsh \symbol{45}l|
| \symbol{124} \symbol{124} \ \ \symbol{92}\symbol{45}\symbol{45}\symbol{45} 30193 jmates sleep 31|
| \symbol{124} \symbol{124}\symbol{45}\symbol{45}= 00531 jmates zsh \symbol{45}l|
| \symbol{124} \symbol{124}\symbol{45}\symbol{45}= 00534 jmates zsh \symbol{45}l|
\end{lstlisting}

As an exercise, consider why the \texttt{sleep 31} but not the
\texttt{sleep 33} appears in the listing. What happens if the
\texttt{echo new} command is placed before the \texttt{sleep 31}? \\

The commands can be better tied to the process table by logging the PID
somewhere; this may be necessary on busy systems where there are many
other processes that may match what is being searched for:

\begin{lstlisting}
| ( echo >\symbol{38}2 \symbol{36}\symbol{36}; sleep 31; ... ) \symbol{124} ...|
\end{lstlisting}

\texttt{\symbol{36}\symbol{36}} however does not identify subshells:

\begin{lstlisting}
|\symbol{36} \textbf{(echo >\symbol{38}2 p\symbol{36}\symbol{36}; echo foo) \symbol{124} while read l; do echo s\symbol{36}\symbol{36}; done}|
|p22531|
|s22531|
\end{lstlisting}

A shell may have variables that will display the ID of the subshell.

\begin{lstlisting}
|\symbol{36} \textbf{exec mksh}|
|\symbol{36} \textbf{echo \symbol{36}BASHPID ; ( echo \symbol{36}BASHPID )}|
|30300|
|30303|
|\symbol{36} \textbf{exec zsh}|
|\symbol{37} \textbf{zmodload zsh/system}|
|\symbol{37} \textbf{print \symbol{36}sysparams\symbol{91}pid\symbol{93}; (print \symbol{36}sysparams\symbol{91}pid\symbol{93})}|
|30294|
|30296|
\end{lstlisting}

Another option may be to change the name of a process to make it easier
to find; various languages also have \texttt{sleep} calls that avoid the
complication of a \texttt{( sleep ... ; command ... )} subshell:

\begin{lstlisting}
|\symbol{37} \textbf{perl \symbol{45}e '\symbol{36}0 = "asdf"; sleep 99' \symbol{38}}|
|\symbol{91}1\symbol{93} 30390|
|\symbol{37} \textbf{ruby \symbol{45}e '\symbol{36}0 = "asdf"; sleep 99' \symbol{38}}|
|\symbol{91}2\symbol{93} 30391|
|\symbol{37} \textbf{ps o pid,command \symbol{124} grep 'asd\symbol{91}f\symbol{93}'}|
|30390 asdf|
|30391 asdf|
\end{lstlisting}

Note that process naming is not portable; the above is from Mac OS X; on
OpenBSD the name includes the name of the interpreter:

\begin{lstlisting}
|\symbol{36} \textbf{perl \symbol{45}e '\symbol{36}0 = "asdf"; sleep 99' \symbol{38}}|
|\symbol{91}1\symbol{93} 15860|
|\symbol{36} \textbf{ruby \symbol{45}e '\symbol{36}0 = "asdf"; sleep 99' \symbol{38}}|
|\symbol{91}2\symbol{93} 52604|
|\symbol{36} \textbf{ps o pid,command \symbol{124} grep 'asd\symbol{91}f\symbol{93}'}|
|15860 perl: asdf (perl)|
|52604 ruby: asdf (ruby30)|
\end{lstlisting}

Where possible use the \texttt{pgrep(1)} or \texttt{pkill(1)} commands
instead of \texttt{ps ... | grep ...}

\begin{lstlisting}
|\symbol{37} \textbf{sleep 99 \symbol{38}}|
|\symbol{91}1\symbol{93} 30439|
|\symbol{37} \textbf{ps o pid,command \symbol{124} grep sleep}|
|30439 sleep 99|
|30441 grep sleep|
|\symbol{37} \textbf{pgrep sleep}|
|30439|
\end{lstlisting}

Process management tools are unportable; review the documentation
specific to the commands for the system in question. A coworker once
wrote a \texttt{killall} command for Digital UNIX, but ended up running
the vendor-provided \texttt{killall} on Solaris when they ran their
script there. Ooops. \\

Another concern is that process listings are only a snapshot; processes
can be created and destroyed while \texttt{ps} is running and thus may
not be seen by \texttt{ps}. \texttt{CVE\symbol{45}2018\symbol{45}1121}
details a trick where a process would receive a notification when the
process table was being scanned and would then fork itself to avoid
being listed in the process table. \\

Process tracing facilities can also track commands; \texttt{acct(2)}
appeared in Version 7 AT\&T UNIX. Since then more elaborate tracing
facilities have been created, notably DTrace, though other tools such as
\texttt{ktrace} or \texttt{strace} or \texttt{sysdig} are useful as
well. On Linux \texttt{strace(1)} collects information on all the system
calls made; the above shell pipeline run under \texttt{strace} reveals
that three unique processes were traced.

\begin{lstlisting}
|\symbol{36} \textbf{cat cmd}|
|\symbol{35}!/bin/sh|
|echo new \symbol{124} while read l; do var=\symbol{36}l; echo "x\symbol{36}var"; done|
|\symbol{36} \textbf{chmod +x cmd}|
|\symbol{36} \textbf{strace \symbol{45}ff \symbol{45}o out ./cmd}|
|xnew|
|\symbol{36} \textbf{echo out*}|
|out.22172 out.22173 out.22174|
\end{lstlisting}

System calls operate at a different level than the shell code; the
\texttt{new} string appears in \texttt{read(2)} and \texttt{write(2)}
system calls

\begin{lstlisting}
|\symbol{36} \textbf{grep new out*}|
|out.22172:read(3, "\symbol{35}!/bin/sh\symbol{92}necho new \symbol{124} while read "...|
|out.22172:read(255, "\symbol{35}!/bin/sh\symbol{92}necho new \symbol{124} while read "...|
|out.22173:write(1, "new\symbol{92}n", 4)                    = 4|
|out.22174:write(1, "xnew\symbol{92}n", 5)                   = 5|
|\symbol{36} \textbf{man 2 read \symbol{124} col \symbol{45}bx \symbol{124} awk '/read\symbol{92}(/\symbol{123}print;exit\symbol{125}'}|
| \ \ \ \ \ \ ssize\symbol{95}t read(int fd, void *buf, size\symbol{95}t count);|
\end{lstlisting}

made by the main shell process, PID 22172, that reads the script
\texttt{cmd} off of the filesystem, and two subshell processes, one that
writes the string (PID 22173, corresponding to the \texttt{echo new}
shell code) and the other that carries out the code for the
\texttt{while} loop (PID 22174, shell code \texttt{echo
"x\symbol{36}var"}). Note that PID 22174 must at some point read in the
string \texttt{new}, written to it by PID 22173\ldots why does that not
appear? An inspection of the trace file reveals single character reads
of the string via standard input (file descriptor \texttt{0}):

\begin{lstlisting}
|\symbol{36} \textbf{grep \symbol{45}A 3 '"n' out.22174}|
|read(0, "n", 1) \ \ \ \ \ \ \ \ \ \ \ \ \ \ \ \ \ \ \ \ \ \ \ \ = 1|
|read(0, "e", 1) \ \ \ \ \ \ \ \ \ \ \ \ \ \ \ \ \ \ \ \ \ \ \ \ = 1|
|read(0, "w", 1) \ \ \ \ \ \ \ \ \ \ \ \ \ \ \ \ \ \ \ \ \ \ \ \ = 1|
|read(0, "\symbol{92}n", 1)                        = 1|
\end{lstlisting}

String and file path tracing with \texttt{strace} may require suitable
\texttt{\symbol{45}s} and \texttt{\symbol{45}y} flags especially when
large amounts of data--where ``large'' is anything longer than 32
characters--or many different files are being used. If this sounds all
very new and strange, ``Advanced Programming in the Unix
Environment''\cite{stevens2013} is a classic text; study the C programs
therein and trace them as they run.

\section*{Exec Wrappers}

Less complicated than tracing and useful for other purposes are small
wrapper programs that carry out some task and then \texttt{exec} another
program. These are typically but need not be shell scripts. Instead of
tracing a huge application, one might have that application call the
following \texttt{exec} wrapper:

\begin{lstlisting}
|\symbol{35}!/bin/sh|
|TMPFILE=`mktemp \symbol{45}t /tmp/log.XXXXXXXXXX` \symbol{38}\symbol{38} \symbol{92}|
| \  ( id ; tty ; env ; echo "\symbol{36}@" ) >\symbol{62} "\symbol{36}TMPFILE"|
|exec /path/to/the/real/program "\symbol{36}@"|
\end{lstlisting}

Then the huge application can be run and the resulting
\texttt{/tmp/log.*} files inspected for the details logged. Targeted
tracing could also be accomplished with \texttt{sysdig} on Linux, among
various other options. \\

The wrapper program should not alter the standard input streams; reading
from standard input would by default prevent what was read from reaching
the real program. \\

A common use for a wrapper is to alter environment variables:

\begin{lstlisting}
|\symbol{36} \textbf{cat wrapper}|
|\symbol{35}!/bin/sh|
|unset SECRET\symbol{95}API\symbol{95}KEY|
|export PATH=/plugin/dir:\symbol{36}PATH|
|exec "\symbol{36}@"|
|\symbol{36} \textbf{chmod +x ./wrapper}|
|\symbol{36} \textbf{export SECRET\symbol{95}API\symbol{95}KEY=muffins\symbol{45}dreamspork\symbol{45}hodgepodge}|
|\symbol{36} \textbf{./wrapper sh \symbol{45}c 'echo ">\symbol{36}SECRET\symbol{95}API\symbol{95}KEY<"'}|
|><|
\end{lstlisting}

The downside of this method is that there is an additional fork and exec
of a process; the environment could instead be customized in the parent
process between the \texttt{fork} and subsequent \texttt{exec}. On the
other hand, the parent process may not be possible to modify, and in
many cases such wrapper programs are not run where performance is
critical. Shell functions can also be used to replace wrapper scripts.
Care must be taken that the shell function does not call itself.

\begin{lstlisting}
|\symbol{35} bad\symbol{33} calls itself|
|function info \symbol{123} info "\symbol{36}@" 2>/dev/null \symbol{124} \symbol{36}PAGER; \symbol{125}|

|\symbol{35} better (emacs users might disagree)|
|function info \symbol{123} command info "\symbol{36}@" 2>/dev/null \symbol{124} \symbol{36}PAGER; \symbol{125}|
\end{lstlisting}

Wrappers may be necessary for security where a process is unable to
properly clean up the environment. For example, one may wish to
ensure that a program is not passed duplicate environment variables.
For this I use the \texttt{nodupenv} wrapper program. Also required
is \texttt{dupenv} to create duplicate environment variables for
testing purposes.

\begin{lstlisting}
|\symbol{36} \textbf{dupenv FOO=x FOO=y env \symbol{124} grep FOO}|
|FOO=x|
|FOO=y|
|\symbol{36} \textbf{dupenv FOO=x FOO=y FOO=z nodupenv env \symbol{124} grep FOO}|
|FOO=x|
\end{lstlisting}

Wait, environment variables can be duplicated? How does that work? Now
might be a good time to talk about programs and their arguments in some
detail, but so ends the shell game.

\bibliography{references}
\end{document}
