\documentclass[10pt,a4paper]{article}
\setlength{\topmargin}{-5mm}
\setlength{\textheight}{244mm}
\setlength{\oddsidemargin}{0mm}
\setlength{\textwidth}{165mm}

\setlength{\parindent}{0pt}

\usepackage{fancyvrb}
\DefineVerbatimEnvironment{verbatim}{Verbatim}{xleftmargin=4mm}

\usepackage{lmodern}
\usepackage[T1]{fontenc}
\usepackage{pxfonts}
\usepackage{textcomp}

\usepackage[square]{natbib}

\usepackage{listings}
\lstset{basicstyle=\ttfamily,
escapeinside={||}}

\frenchspacing

\title{A \texttt{fork()} in the Road}
\author{Jeremy Mates}
\date{June 27, 2021}

\usepackage{hyperref}
\hypersetup{pdfauthor={Jeremy Mates},pdftitle={A fork() in the Road}}

\begin{document}
\bibliographystyle{plainnat}
\maketitle

In ``A \texttt{fork()} in the road''\cite{Hotos2019} two programs are
mentioned as being slow with with \texttt{fork(2)}--Chrome, and
\texttt{node}. These are atypical choices, as we shall see. Chrome uses
abnormal amounts of memory, and a good security policy might well deny
Chrome the right to fork if, hypothetically, Chrome were to suffer from
some security issue. What should Chrome do if it cannot fork? Perhaps it
should, at launch, fork a small supervisor process--maybe called
\texttt{init}--which would then be responsible for starting the other
processes that Chrome needs--perhaps via some ``system daemon''
framework. This way, processes with heavy memory use or multithreading
needs would only do that, and would not ever be calling fork or exec.
Did I just describe an operating system? Probably. Is Chrome a not very
good operating system? Probably. \\

\texttt{node} used to crash (in 2019, but no longer does in
\texttt{v12.16.1}) if you printed to the console in a loop via

\begin{lstlisting}
|\symbol{36} \textbf{node -e 'while (1) \{ process.stdout.write("a") \}' > /dev/null}|
|Bus error (core dumped)|
\end{lstlisting}

so it's not surprising to me that it would have other problems such as
taking a long time to fork. Perhaps as skill with heavily
multithreaded programs that actually need to fork goes up such issues
will get ironed out? \\

Anyways, how atypical is Chrome? Let's look at top memory use on the
2009 MacBook.

\begin{lstlisting}
|PID \ COMMAND \ \ \ \ \ \symbol{37}CPU TIME \ \ \ \ \symbol{35}TH \ \ \symbol{35}WQ \ \symbol{35}PORT MEM|
|0 \ \ \ kernel\symbol{95}task \ 0.7 \ 12:14.54 114/2 0 \ \ \ 2 \ \ \ \ 380M|
|380 \ QuickTime Pl 0.0 \ 50:45.86 15 \ \ \ 2 \ \ \ 342 \ \ 57M|
|232 \ WindowServer 0.3 \ 47:39.50 5 \ \ \ \ 2 \ \ \ 311 \ \ 56M|
|281 \ iTerm2 \ \ \ \ \ \ 0.2 \ 03:04.65 6 \ \ \ \ 0 \ \ \ 252 \ \ 28M|
|287 \ Finder \ \ \ \ \ \ 0.0 \ 00:07.63 3 \ \ \ \ 0 \ \ \ 271 \ \ 26M|
|285 \ Dock \ \ \ \ \ \ \ \ 0.0 \ 00:01.76 3 \ \ \ \ 0 \ \ \ 225 \ \ 24M|
|463 \ Sudoku \ \ \ \ \ \ 0.0 \ 00:09.48 6 \ \ \ \ 1 \ \ \ 275 \ \ 22M|
|482 \ Dictionary \ \ 0.0 \ 00:00.79 5 \ \ \ \ 0 \ \ \ 200 \ \ 19M|
\end{lstlisting}

Do those memory intensive processes fork? At all? What does fork a lot?
The shells, which are using\ldots oh there they are, 1888K, 2268K, etc.
Would \texttt{posix\_spawn} even help for processes that small, assuming
you can live without the error reporting? How much memory does Chrome
use? I'd have to install it to find out; Safari.app under Mac OS X 10.11
uses around 128M doing nothing. I've heard stories of Chrome eating into
another order of magnitude, or two? Chrome having problems with
\texttt{fork()} is less about \texttt{fork()} and more about Chrome
being, well, bloaty.\footnote{Common references here include ``Mr.
Creosote'' from ``The Meaning of Life'' or the ``Akira'' stadium scene
to describe how a 1990s hypermedia platform has metastasized. Alas,
computer science has not granted us laser rifles nor wafer-thin
after-dinner mints.}

\section*{Cherry Picking}

I too can cherry pick processes favorable to my position. Here is what
forks a lot on my OpenBSD system:

\begin{lstlisting}
|\symbol{36} \textbf{ps axo rss,vsz,args \symbol{124} grep s\symbol{91}h\symbol{93}}|
|968 \ 1008 /bin/sh /etc/X11/xenodm/Xsession|
|1148 \ 1220 /bin/ksh \symbol{45}l|
|1052 \ 1020 /bin/ksh \symbol{45}l|
|1068 \ 1152 /bin/ksh \symbol{45}l|
|1036 \ 1016 /bin/ksh \symbol{45}l|
\end{lstlisting}

And next we sort on the \texttt{RES} column in \texttt{top(1)} with the
output cleaned up a bit:

\begin{lstlisting}
| \ \ PID USERNAME SIZE \ \ RES STATE \ \ TIME \ \ CPU COMMAND|
| 64699 \symbol{95}x11 \ \ \ \ \ 42M \ \ 52M sleep/0 0:19 0.49\symbol{37} Xorg|
| 54110 jmates \ \ \ 19M \ \ 24M sleep/0 0:02 0.00\symbol{37} mupdf\symbol{45}x11|
| 49777 jmates \ \ \ 12M \ \ 18M sleep/0 0:04 0.00\symbol{37} irssi|
| \ 2134 root \ \ \ 2100K 8080K idle \ \ \ 0:01 0.00\symbol{37} xenodm|
| \ 3606 jmates \ 1148K 6748K sleep/1 0:00 0.00\symbol{37} cwm|
| 45572 jmates \ 1884K 5784K sleep/1 0:05 0.00\symbol{37} xterm|
| 14369 jmates \ 1884K 5756K sleep/1 0:07 0.00\symbol{37} xterm|
| 86394 jmates \ 1868K 5740K sleep/1 0:01 0.00\symbol{37} xterm|
| 46500 jmates \ 1864K 5740K sleep/0 0:01 0.00\symbol{37} xterm|
| 76013 jmates \ 4812K 5344K sleep/1 0:01 0.00\symbol{37} vim|
\end{lstlisting}

\texttt{cwm(1)} and \texttt{vim(1)} are the only where notable fork/exec
would happen, and not ever in any sort of performance critical path;
worst case would be the a fork/exec of a script that in turn forks/execs
four \texttt{xterm} which in turn forks/execs four scripts that start or
resume four \texttt{tmux} sessions. This all happens quickly enough to
not bother me; Chromium, meanwhile, back when I had it installed, I
thought it had wedged at startup as it was burning through CPU but not
apparently doing anything.\footnote{Microsoft Teams I started exactly
once on the 2009 MacBook. Then I uninstalled it. It used more CPU just
getting up off disk than the IRC client had in days of running.}
Patience, how long will that take? In the other processes fork/exec
would be minimal to zero; in the case of Xorg and irssi
\texttt{execve(2)} is not even allowed. And the multi-threaded programs,
all of them.

\begin{lstlisting}
| 49777 irssi|
| 64699 /usr/X11R6/bin/X|
\end{lstlisting}

\section*{The Replacements}

So what are we to replace \texttt{fork} with? The paper indicates
\texttt{posix\_spawn}, which is ``not a complete replacement for fork
and exec\ldots it\ldots lacks an effective error reporting
mechanism''\citep[p.5]{Hotos2019}. Perhaps someone could spend time
(isn't programmer time expensive?) to catalog what existing uses of fork
could actually be rewritten--and then debugged, and tests written for
any regressions (wait, are there even tests?), the documentation
updated, etc.--to use \texttt{posix\_spawn} and then process startup
might be a bit faster? Is this even a problem? \texttt{vim(1)} is slow
to come up cold off of disk, but that's mostly due to the spinning metal
hard drive and the 0.8 GHz CPU running OpenBSD, an OS well noted for its
speed. Here are some hot cache forks and execs:

\begin{lstlisting}
|\symbol{36} \textbf{repeat 3 time vim \symbol{45}c quit}|
| \ \ \ \ \ \ \ 0.11 real \ \ \ \ \ \ \ \ 0.08 user \ \ \ \ \ \ \ \ 0.03 sys|
| \ \ \ \ \ \ \ 0.11 real \ \ \ \ \ \ \ \ 0.09 user \ \ \ \ \ \ \ \ 0.02 sys|
| \ \ \ \ \ \ \ 0.11 real \ \ \ \ \ \ \ \ 0.08 user \ \ \ \ \ \ \ \ 0.03 sys|
|\symbol{36} \textbf{highcpu}|
|\symbol{36} \textbf{repeat 3 time vim \symbol{45}c quit}|
| \ \ \ \ \ \ \ 0.04 real \ \ \ \ \ \ \ \ 0.04 user \ \ \ \ \ \ \ \ 0.01 sys|
| \ \ \ \ \ \ \ 0.04 real \ \ \ \ \ \ \ \ 0.03 user \ \ \ \ \ \ \ \ 0.01 sys|
| \ \ \ \ \ \ \ 0.05 real \ \ \ \ \ \ \ \ 0.04 user \ \ \ \ \ \ \ \ 0.01 sys|
\end{lstlisting}

How would \texttt{posix\_spawn} help here? A migration is a non-starter
if you need that error reporting, or otherwise looks to be of moderate
effort and low reward. OpenBSD shows zero use of \texttt{posix\_spawn}
outside of LLVM and libc under \texttt{/usr/src}. They probably have
more important things to work on. I first heard of \texttt{posix\_spawn}
during a job interview. The question came from a Windows systems
administrator. Perhaps if \texttt{vim(1)} were being spawned often from
software with excessive memory use? But why run such? Why not launch
\texttt{vim(1)} from a small, fast shell?

\subsection*{An Aside with \texttt{nmap}}

I did once have a problem with launching many \texttt{nmap(1)}
processes; I was testing a LDAP server, where lots of connections (or
traffic, or something) would (sometimes) wedge the server--obviously
only in production, and never in the test environment. The test script
probably looked something like:

\begin{lstlisting}
|for i in ...; do nmap ... \symbol{38} done|
\end{lstlisting}

Except it was much less pretty, maybe with a \texttt{wait} and a second
loop to keep spawning more nmaps. This obviously was a lot of forks and
obviously was quite slow. Did I rewrite this slow thing to use
\texttt{posix\_spawn}? Nope. I wrote a new script--two actually--that
performed a test inside a very tight loop in a single process. The shell
forked and execed these test programs, and LDAP fell over. Success!
Where would \texttt{posix\_spawn} help? Maybe to shave off some script
startup time? Noise, compared to converting a quick (but slow to run)
``I have an idea\ldots'' shell one-liner--the prototype--into a pair of
very fast test scripts--a first implementation.

\subsection*{More Alternatives}

vfork(2) is mentioned as an alternative BSD thing. Let's read the
fine manual.

\begin{lstlisting}
  BUGS
       This system call will be eliminated when proper system sharing
       mechanisms are implemented. Users should not depend on the memory
       sharing semantics of vfork as it will, in that case, be made
       synonymous to fork.
\end{lstlisting}

Based on that I'd agree with the article that ``in most cases it is
better avoided''\citep[p.5]{Hotos2019}. The documentation on Mac OS X
10.11 is not known for being updated or up-to-date; let's try OpenBSD.

\begin{lstlisting}
  DESCRIPTION
       vfork() was originally used to create new processes without fully
       copying the address space of the old process, which is
       horrendously inefficient in a paged environment. It was useful
       when the purpose of fork(2) would have been to create a new
       system context for an execve(2). Since fork(2) is now efficient,
       even in the above case, the need for vfork() has diminished.
  ...
  HISTORY
       The vfork() function call appeared in 3.0BSD with the additional
       semantics that the child process ran in the memory of the parent
       until it called execve(2) or exited. That sharing of memory was
       removed in 4.4BSD, leaving just the semantics of blocking the
       parent until the child calls execve(2) or exits.
\end{lstlisting}

Not seeing anything new nor viable about \texttt{vfork(2)}. Why is it
even listed as an alternative? \\

The article then mentions some low-level something from elsewhere,
and concludes that it ``seems at first glance challenging, but may
also be productive for future research''\citep[p.6]{Hotos2019}. Good
luck with that? \\

How about \texttt{clone()}?

\begin{lstlisting}
% |\textbf{man clone}|
No manual entry for clone

$ |\textbf{man clone}|
man: No entry for clone in the manual.
\end{lstlisting}

So not portable, plus ``clone suffers most of the same
problems as fork''\citep[p.6]{Hotos2019}. \\

To conclude, there are simply not any viable alternatives; the call to
``strongly discourage'' the use of fork in new code is both laughable
and absurd.\footnote{Or is FUD from a company known for spreading
that\ldots} Why not simply use \texttt{fork()} in a small and fast
single threaded processes\ldots oh wait, the authors did realize that.
In hindsight.

\bibliography{references}
\end{document}
